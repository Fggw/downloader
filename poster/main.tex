%%%%%%%%%%%%%%%%%%%%%%%%%%%%%%%%%%%%%%%%%
% Jacobs Landscape Poster
% LaTeX Template
% Version 1.0 (29/03/13)
%
% Created by:
% Computational Physics and Biophysics Group, Jacobs University
% https://teamwork.jacobs-university.de:8443/confluence/display/CoPandBiG/LaTeX+Poster
% 
% Further modified by:
% Nathaniel Johnston (nathaniel@njohnston.ca)
%
% This template has been downloaded from:
% http://www.LaTeXTemplates.com
%
% License:
% CC BY-NC-SA 3.0 (http://creativecommons.org/licenses/by-nc-sa/3.0/)
%
%%%%%%%%%%%%%%%%%%%%%%%%%%%%%%%%%%%%%%%%%

%----------------------------------------------------------------------------------------
%	PACKAGES AND OTHER DOCUMENT CONFIGURATIONS
%----------------------------------------------------------------------------------------

\documentclass[final]{beamer}

\usepackage[scale=1.24]{beamerposter} % Use the beamerposter package for laying out the poster
\usepackage{subfigure}

\usetheme{confposter} % Use the confposter theme supplied with this template

\setbeamercolor{block title}{fg=ngreen,bg=white} % Colors of the block titles
\setbeamercolor{block body}{fg=black,bg=white} % Colors of the body of blocks
\setbeamercolor{block alerted title}{fg=white,bg=dblue!70} % Colors of the highlighted block titles
\setbeamercolor{block alerted body}{fg=black,bg=dblue!10} % Colors of the body of highlighted blocks
% Many more colors are available for use in beamerthemeconfposter.sty

%-----------------------------------------------------------
% Define the column widths and overall poster size
% To set effective sepwid, onecolwid and twocolwid values, first choose how many columns you want and how much separation you want between columns
% In this template, the separation width chosen is 0.024 of the paper width and a 4-column layout
% onecolwid should therefore be (1-(# of columns+1)*sepwid)/# of columns e.g. (1-(4+1)*0.024)/4 = 0.22
% Set twocolwid to be (2*onecolwid)+sepwid = 0.464
% Set threecolwid to be (3*onecolwid)+2*sepwid = 0.708

\newlength{\sepwid}
\newlength{\onecolwid}
\newlength{\twocolwid}
\newlength{\threecolwid}
\setlength{\paperwidth}{48in} % A0 width: 46.8in
\setlength{\paperheight}{36in} % A0 height: 33.1in
\setlength{\sepwid}{0.024\paperwidth} % Separation width (white space) between columns
\setlength{\onecolwid}{0.22\paperwidth} % Width of one column
\setlength{\twocolwid}{0.464\paperwidth} % Width of two columns
\setlength{\threecolwid}{0.708\paperwidth} % Width of three columns
\setlength{\topmargin}{-0.5in} % Reduce the top margin size
%-----------------------------------------------------------

\usepackage{graphicx}  % Required for including images

\usepackage{booktabs} % Top and bottom rules for tables

%----------------------------------------------------------------------------------------
%	TITLE SECTION 
%----------------------------------------------------------------------------------------

\title{Predicting Food Inspection Outcomes in Chicago} % Poster title

\author{Luke Farewell, Jake Gober, Sam Green, Jeremy Welborn} % Author(s)

\institute{Computer Science 109a / Statistics 121a, Harvard University} % Institution(s)

% \logo{figures/SEASLogo_CMYK_Centered.png}

%----------------------------------------------------------------------------------------

\begin{document}

\addtobeamertemplate{block end}{}{\vspace*{2ex}} % White space under blocks
\addtobeamertemplate{block alerted end}{}{\vspace*{2ex}} % White space under highlighted (alert) blocks

\setlength{\belowcaptionskip}{2ex} % White space under figures
\setlength\belowdisplayshortskip{2ex} % White space under equations

\begin{frame}[t] % The whole poster is enclosed in one beamer frame

\begin{columns}[t] % The whole poster consists of three major columns, the second of which is split into two columns twice - the [t] option aligns each column's content to the top

\begin{column}{\sepwid}\end{column} % Empty spacer column

\begin{column}{\onecolwid} % The first column

%----------------------------------------------------------------------------------------
%	OBJECTIVES
%----------------------------------------------------------------------------------------

\begin{alertblock}{Objectives}

Chicago's Department of Public Health is responsible for 
inspecting 15,000 food establishments across the city. 
Our goal was to reduce the amount of time required to 
discover critical violations: 

\begin{itemize}
\item Aggregate and clean useful data sources.
\item Build models for probabilities of failure.
\item Optimize cost of inspection by recommending inspections
in a probability-ranked order.
\end{itemize}

\end{alertblock}

%----------------------------------------------------------------------------------------
%	INTRODUCTION
%----------------------------------------------------------------------------------------

\begin{block}{Data: Sources and Cleaning}

% \setlength\tabcolsep{1.2in}
\def\arraystretch{1.5}
\begin{tabular}{|c| c|}
\hline
\textbf{Food Inspections} & \textbf{Business Activities} \\ \hline
\textbf{Weather} & \textbf{Crime Reports} \\ \hline
\textbf{Sanitation Complaints} & \textbf{Business Location}\\ \hline
\end{tabular}\vspace{0.2in.}
Sources: City of Chicago Open Data Portal, NOAA. 

\begin{enumerate}
    \item Collate inspections with businesses for history.
    \item Process free text columns.
    \item Bucket spatial data with a grid (see visualizations).
\end{enumerate}

\end{block}

\vspace{-1in.}
\begin{block}{Exploration (I)}
\begin{figure}
\includegraphics[width=\linewidth]{figures/luke_placeholder.png}
\caption{Scoring Function}
\end{figure}
\end{block}


%----------------------------------------------------------------------------------------

\end{column} % End of the first column

\begin{column}{\sepwid}\end{column} % Empty spacer column

\begin{column}{\twocolwid} % Begin a column which is two columns wide (column 2)


%%% --- Pull up the alert --- %%%

\begin{columns}[t, totalwidth=\twocolwid]

\begin{column}{\onecolwid}\vspace{-.6in.}
\begin{block}{Exploration (II)}
Model success curve 1.
\begin{figure}
\includegraphics[width=\linewidth]{placeholder.jpg}
\caption{Scoring Function}
\end{figure}
\end{block}

\end{column}

\begin{column}{\onecolwid}\vspace{-.6in.}
\begin{block}{Results (I)}
Model success curve 2.
\begin{figure}
\includegraphics[width=\linewidth]{placeholder.jpg}
\caption{Scoring Function}
\end{figure}
\end{block}
\end{column}

\end{columns}

\begin{alertblock}{Choosing a Scoring Function}

To measure \textit{rank accuracy} and match the 
recommendation setting, 
we selected \textbf{log loss} as our scoring function.

\begin{equation}
-\frac{1}{n} \sum_{1}^{n} [y_i \log(p_i)-(1-y_i)\log(1-p_i)]
\end{equation}
for $n$ observations, where the $i$th observation is of correct class $y_i \in \{0,1\}$ 
which our model assigns probability $p_i$.

\end{alertblock}

\begin{column}{\twocolwid}\vspace{-.4in} % The first column within column 2 (column 2.1)

%----------------------------------------------------------------------------------------
%   Data Collection
%----------------------------------------------------------------------------------------

\begin{block}{Exploration (III): Spatial Predictors \& Neighborhood Dynamics}

\begin{figure}
% \subfigure[] \includegraphics[width=0.8\linewidth]{figures/crime.png}
%
\includegraphics[width=\linewidth]{figures/both.pdf}
\caption{Crime (left) and Sanitation Complaint (Right) by Location. Darker shades represent 
a larger fraction of observations. }
\end{figure}

\end{block}
\end{column}

\end{column} % End of the second column

\begin{column}{\sepwid}\end{column} % Empty spacer column

\begin{column}{\onecolwid} % The third column

%----------------------------------------------------------------------------------------
%	CONCLUSION
%----------------------------------------------------------------------------------------

\begin{block}{Results (II) \& Discussion}

Nunc tempus venenatis facilisis. \textbf{Curabitur suscipit} consequat eros non porttitor. Sed a massa dolor, id ornare enim. Fusce quis massa dictum tortor \textbf{tincidunt mattis}. Donec quam est, lobortis quis pretium at, laoreet scelerisque lacus. Nam quis odio enim, in molestie libero. Vivamus cursus mi at \textit{nulla elementum sollicitudin}.

Maecenas ultricies feugiat velit non mattis. Fusce tempus arcu id ligula varius dictum. 
\begin{itemize}
\item Curabitur pellentesque dignissim
\item Eu facilisis est tempus quis
\item Duis porta consequat lorem
\end{itemize}

\begin{figure}
\includegraphics[width=\linewidth]{placeholder.jpg}
\caption{This is a result curve.}
\end{figure}

\end{block}

\vspace{-1in.}
\begin{block}{Next Steps}

LATIN LATIN LATIN LATIN
LATIN LATIN LATIN LATIN
LATIN LATIN LATIN LATIN
LATIN LATIN LATIN LATIN

\end{block}


%----------------------------------------------------------------------------------------
%	ACKNOWLEDGEMENTS
%----------------------------------------------------------------------------------------

% \setbeamercolor{block title}{fg=red,bg=white} % Change the block title color

% \begin{block}{Acknowledgements}

% \small{\rmfamily{Nam mollis tristique neque eu luctus. Suspendisse rutrum congue nisi sed convallis. Aenean id neque dolor. Pellentesque habitant morbi tristique senectus et netus et malesuada fames ac turpis egestas.}} \\

% \end{block}

%----------------------------------------------------------------------------------------
%	CONTACT INFORMATION
%----------------------------------------------------------------------------------------

\setbeamercolor{block alerted title}{fg=black,bg=norange} % Change the alert block title colors
\setbeamercolor{block alerted body}{fg=black,bg=white} % Change the alert block body colors

\begin{alertblock}{Contact Information}

\begin{itemize}
\item Web: \href{https://fggw.github.io/foodinspections/}{fggw.github.io/foodinspections/}
\item Code: \href{http://github.com/fggw/foodinspections}{github.com/fggw/foodinspections/}
\item Contact: \{lfarewell, jgober, samuelgreen, jeremywelborn\}@college.harvard.edu
\end{itemize}

\end{alertblock}

%----------------------------------------------------------------------------------------
%   REFERENCES
%----------------------------------------------------------------------------------------

\begin{block}{References}
\begin{itemize}
    \item \href{https://chicago.github.io/food-inspections-evaluation/}{https://chicago.github.io/food-inspections-evaluation/}
\end{itemize}
\end{block}

%----------------------------------------------------------------------------------------

\end{column} % End of the third column

\end{columns} % End of all the columns in the poster

\end{frame} % End of the enclosing frame

\end{document}
